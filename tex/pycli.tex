\documentclass[12pt]{article} % use larger type; default would be 10pt

%packages
\usepackage[utf8]{inputenc} % set input encoding (not needed with XeLaTeX)
\usepackage{fancyhdr}
\usepackage{float}
\usepackage{geometry}
\usepackage{ulem}
\usepackage{soul}
\usepackage{color}
\usepackage{graphicx}
\usepackage{hyperref}
\usepackage{array}
\usepackage{caption}
\usepackage{titling}
\usepackage{enumerate} 
\usepackage[dvipsnames]{xcolor}
\usepackage{amsmath}
\usepackage{amssymb}
\usepackage[compact]{titlesec}


 %put box around figure captions
\makeatletter
\long\def\@makecaption#1#2{%
  \vskip\abovecaptionskip
  \sbox\@tempboxa{\fbox{#1: #2}}%
  \ifdim \wd\@tempboxa >\hsize
    \fbox{\parbox{\dimexpr\linewidth-2\fboxsep-2\fboxrule}{#1: #2}}\par
  \else
    \global \@minipagefalse
    \hb@xt@\hsize{\hfil\box\@tempboxa\hfil}%
  \fi
  \vskip\belowcaptionskip}
\makeatother

%reduce space between sections
\titlespacing{\section}{0pt}{*1}{*0}
\titlespacing{\subsection}{0pt}{*1}{*0}
\titlespacing{\subsubsection}{0pt}{*0}{*0}


%no indent and modify distance between paragraphs
\setlength\parindent{0pt}
\setlength\parskip{12pt}

%set margins and line spacing
\geometry{margin=1in}
\linespread{1.2}
\geometry{letterpaper}

%math operators
\DeclareMathOperator{\E}{\mathbb{E}}

%set up header and page numbering
\pagestyle{fancy}
\lhead{Scientific Appendix}
\rhead{Timothy Liu}
\pagenumbering{arabic}

\hypersetup{  %set up url
    colorlinks=true,
    linkcolor=blue,
    filecolor=magenta,      
    urlcolor=cyan,
}



\title{PyCli Modeling}
\author{Timothy Liu}

\begin{document}

\maketitle

This document is used to write the equations for the PyCli README.

The solar radiation falling on the entire Earth is:

$$E_{in} = W_{sun} \pi r_{Earth}^2$$

Consider the Earth as a flat disc perpendicular to the direction of the sun. The sunlight falling on a horizontal strip is:

$$E_{in} = 2 W_{sun} \int_{x_0}^{x_f} \sqrt{r_{E}^2 - x^2} dx$$


where $x_0$ and $x_f$ are the distances from the equator of the bottom and top of the strip. Integrating to the get the closed form:

$$E_{in} = 2 W_{sun} \frac{1}{2} x \sqrt{r_{E}^2 - x^2} - \frac{1}{2} r_{E}^2 \tan^{-1}\bigg({\frac{x \sqrt{r_{E}^2 - x^2}}{x^2 - r_{E}^2 }}\bigg)\bigg]_{x_0}^{x_f}$$

$$E_{in} = W_{sun} x \sqrt{r_{E}^2 - x^2} -  r_{E}^2 \tan^{-1}\bigg({\frac{x \sqrt{r_{E}^2 - x^2}}{x^2 - r_{E}^2 }}\bigg)\bigg]_{x_0}^{x_f}$$

However, the boundaries between the cells are defined by lines of latitude rather than as a distance from the equator of a flat disc. The substitution: 

$$x = r_{E} sin(\theta) $$

where $\theta$ is the line of latitude, is used to convert between $x$ and the line of latitude $\theta$. The incoming solar flux per
unit area along this strip is the above result divided by the surface area of the strip on a 3D sphere:

$$A_{strip} = \int_{\theta_{0}}^{\theta_{f}} 2 \pi r_{E} \cos{\theta} r d\theta $$

where $2 \pi r_{E} \cos{\theta}$ is the circumference of the earth at a certain latitude and $r d\theta$ is the north-south distance of a differential along the surface of the earth. This can be simplified to:

$$A_{strip} =  2 \pi r_{E}^2 \sin{\theta} $$

The average solar flux falling on a cell is:

$$E_{in} = \frac{flux_{total}}{A_{strip} n_{cells}}$$

$$\boxed{E_{in} = W_{sun}\frac{x \sqrt{r_{E}^2 - x^2} -  r_{E}^2 \tan^{-1}\bigg({\frac{x \sqrt{r_{E}^2 - x^2}}{x^2 - r_{E}^2 }}\bigg)\bigg]_{x_0}^{x_f}}{2n \pi r_{E}^2 \sin{\theta}}}$$



\end{document}
